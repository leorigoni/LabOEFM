\\subsection{tetamin}
    Come secondo punto del'esperiemento è stato misurato l'angolo di deviazione minima ($\theta_{min}$) per ciascuna delle bande di emissione del Mercurio (Mg).\\
    È stato, dopo aver rimosso il prisma dalla piattaforma, come prima cosa allineato il cannocchinale al collimatore in maniera che si trovasssero paralleli tra di loro, 
    così da poter misurare l'angolo a zero $\theta_{0}$ successivamente, inserito nuovamente il prisma sulla piattaforma, sono stati cercati i raggi delle bande di emissione 
    rifratti dal prisma stesso.\\
    A questo punto si è ricercato $\theta_{min}$, per la prima banda si emissione muovendo in un primo momento, sia la piattafoma del prisma che la piattaforma 
    del cannocchiale, in maniera grossolana e successivamente passando all'utilizzo del nel nonio l'angolo esatto in cui il movimento del raggio rifratto passava dal 
    muoversi in senso antiorario a senso orario o viceversa, riportato questo secondo angolo $\theta_{1}$, per ricavare l'angolo di deviazione minima basterà applicare 
    la seguente formula:\\

    $\theta_{min}=\theta_{1}-\theta_{0}$\\

    È stata poi ripetuta la stessa misura per le restanti bande di emissione per trovarsi i diversi $\theta_{min}$.\\
