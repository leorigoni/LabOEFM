\subsection{Presa dei dati}
    Completata la preparazione della strumentazione, 
    è stato inizialmente orientato l'apparato facendo in modo che il campo magnetico prodotto dalle bobine di Helmholtz, al centro delle stesse, fosse perpendicolare 
    al campo magnetico terrestre, successivamente sono state accese le due differenze di potenziale e il generatore di corrente.\\
    A questo punto sono state prese le prime misure, si è scelto di mantenere il diametro del \textbf{FASCIO} di elettroni fisso, in maniera che per prendere le misure bisognasse 
    solamente modificare l'intensità di corrente successivamente la differenza di potenziale, tentando di riallineare il \textbf{FASCIO} alle due barre che ne segnavano il diametro, 
    o viceversa, sono state quindi prese dieci coppie di misure d'intensità di corrente e differenza di potenziale per il dato \textbf{FASCIO}.\\
    Successivamente si è modificato il \textbf{FASCIO} e si sono prese ulteriori dieci misure.\\
    Il procedimento sopra spiegato è stato successivamente ripetuto allineando il campo magnetico prodotto delle bobine, prima in maniera che fosse parallelo al campo 
    magnetico terrestre e successivamente che fosse antiparallelo.